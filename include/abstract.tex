\chapter*{ABSTRACT}
\begin{singlespace}
    \noindent
    Lingga Rohadyan, 10120594 \\ \\
    \textbf{\textit{IDENTIFICATION OF NUTRITIONAL VALUE IN POPULAR FOOD IN INDONESIA BASED ON IMAGE USING CNN WITH EFFICIENTNETV2 ARCHITECTURE}}\\
    \textit{Thesis, Department of Information Systems, Faculty of Computer Science and Information Technology, Gunadarma University, 2024.}\\
    \textit{Keyword : Convolutional Neural Network, Transfer Learning, Classification.}\\
    (xiv + 98 + \textit{Attachment}) \\ \\
    \textit{Technology is developing very rapidly and has a great impact on human life and society. One of the most interesting aspects of current technological developments is Artificial Intelligence (AI). In development of AI, one of the most influential branches is Machine Learning (ML). One of the applications of ML in the health sector is by using image recognition technology to classify food and estimate its nutritional content. One of the ML techniques for image analysis is Convolutional Neural Network (CNN). This research aims to produce a deep learning model based on the EfficientNetV2 architecture that can classify 10 types of food images and determine their nutritional value. From this research, it can be concluded that the model has successfully classified food images and displayed nutritional estimation. The DGX A100 HPC is proven to be able to perform model training 2x faster than the T4 GPU. The model trained using GPU T4 with the model trained using HPC DGX A100 have similar performance and accuracy which are not significantly different. Evaluation of GPU T4 model managed to get an accuracy of 87.50\% on validation data with average precision, recall, and F1-Score of 88\%. While evaluation of HPC DGX A100 model managed to get an accuracy of 87.25\% on the validation data with precision of 87\%, recall 88\%, and F1-Score 87\%.}  \\ \\
    \textit{Bibliography} (2017 - 2024)
\end{singlespace}
\pagebreak