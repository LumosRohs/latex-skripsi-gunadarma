\chapter{TINJAUAN PUSTAKA}

\section{Artificial Intelligence}
\textit{Artificial Intelligence} (AI) atau kecerdasan buatan merupakan bidang ilmu komputer yang bertujuan untuk mengembangkan mesin atau program komputer yang dalam melakukan tugas yang biasanya memerlukan kecerdasan manusia, seperti pengenalan wajah, pengenalan suara, bahasa alami, analisis data, dan pengambilan keputusan. Kecerdasan buatan mencakup berbagai teknologi seperti \textit{machine learning}, \textit{deep learning}, \textit{natural language processing}, \textit{image processing}, dan \textit{robotics}.

Menurut \cite{abbas2021} Kecerdasan Buatan adalah fenomena sosial dan kognitif fenomena yang memungkinkan mesin untuk berintegrasi secara sosial dengan masyarakat untuk melakukan tugas-tugas kompetitif yang membutuhkan proses kognitif dan berkomunikasi dengan entitas lain dalam masyarakat dengan mengubah pesan dengan konten informasi yang tinggi dan lebih pendek.
