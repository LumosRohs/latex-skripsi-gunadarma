\thispagestyle{empty}
\begin{singlespace}
    \begin{center}
    \fontsize{14}{16}\selectfont
    \textbf{UNIVERSITAS GUNADARMA}

    \vspace{0.25cm}
    
    \fontsize{14}{16}\selectfont
    \textbf{FAKULTAS ILMU KOMPUTER \& TEKNOLOGI INFORMASI}

    \vspace{0.2cm}
    
    \begin{figure}[h]
    \begin{center}\includegraphics[scale=1.5]{images/logo-gunadarma.png}
    \end{center}
    \end{figure}

    \vspace{0.2cm}
    
    \fontsize{14}{16}\selectfont
    \textbf{IDENTIFIKASI NILAI NUTRISI PADA MAKANAN POPULER DI INDONESIA BERBASIS CITRA MENGGUNAKAN CNN DENGAN ARSITEKTUR EFFICIENTNETV2}
    
    \end{center}
    %\vfill
    \vspace{0.75cm}
    
    \begin{center}
    \bfseries
    {Disusun Oleh $:$}
    
    \vspace{0.5cm}
    \begin{tabular}{ll}
    Nama& : Lingga Rohadyan\tabularnewline NPM& : 10120594
    \tabularnewline Program Studi& : Sistem Informasi\tabularnewline
    Pembimbing& : Prof. Dr. Eri Prasetyo Wibowo, SSi.\tabularnewline
    \end{tabular}
    \end{center}
    \vspace{0.75cm}
    \begin{center}
    \bfseries
    Diajukan Guna Melengkapi Sebagian Syarat \\
    Dalam Mencapai Gelar Sarjana Strata Satu (S1)\\
    
    \vspace{1.5cm}
    %\vfill
    
    JAKARTA\\
    2024 %DIGANTI TAHUN PENULISAN SKRIPSI
    
    \end{center}
    %\vfill
\end{singlespace}

\pagebreak