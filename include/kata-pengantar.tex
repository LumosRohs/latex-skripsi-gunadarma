\chapter*{KATA PENGANTAR}

Puji syukur kehadirat Allah SWT atas rahmat dan karunia nya, penulis dapat menyelesaikan tugas akhir yang berjudul “Identifikasi Nilai Nutrisi pada Makanan Populer di Indonesia Berbasis Citra menggunakan CNN dengan arsitektur EfficientNetV2” dengan baik serta sesuai dengan harapan penulis.

Tugas akhir ini disusun sebagai bagian dari syarat dalam mencapai gelar Sarjana Strata Satu pada Jurusan Sistem Informasi, Fakultas Ilmu Komputer dan Teknologi Informasi, Universitas Gunadarma. Dalam proses menyusun tugas akhir ini tentunya tidak mudah bagi penulis, namun berkat bantuan serta kerjasama dari berbagai pihak yang penulis hormati membuat penulisan tugas akhir ini dapat terselesaikan. Untuk itu penulis mengucapkan terima kasih kepada pihak-pihak terkait, diantaranya adalah:

\begin{enumerate}
    \item Prof. Dr. E.S. Margianti, SE., MM., selaku Rektor Universitas Gunadarma.
    \item Prof. Dr. Rer. Nat, Achmad Benny Mutiara, SSi., SKom., selaku Dekan Fakultas Ilmu Komputer dan Teknologi Informasi Universitas Gunadarma.
    \item Dr. Setia Wirawan, SKom., MMSI., selaku Kepala Program Studi Sistem Informasi Universitas Gunadarma.
    \item Dr. Edi Sukirman, SSi., MM., M.I.Kom. selaku Kepala Bagian Sidang Ujian Universitas Gunadarma.
    \item Prof. Dr. Eri Prasetyo Wibowo, SSi. selaku Dosen Pembimbing yang telah memberikan arahan dan waktunya kepada penulis selama penulisan ini berlangsung hingga selesai.
    \item Bapak dan Ibu Dosen Universitas Gunadarma yang telah memberikan ilmu pengetahuan kepada penulis.
\end{enumerate}

Tentunya penulisan ini jauh dari kata sempurna, namun penulis berharap penulisan ini dapat memberikan manfaat dan menambah wawasan bagi pihak-pihak yang membutuhkan. Dengan segala rasa hormat penulis berharap mendapatkan kritik dan saran yang sifatnya membangun untuk menjadi perbaikan di masa yang akan datang.

%\vspace{1cm}
%\begin{flushright}
%    Depok, 22 Agustus 2024\\
%    \begin{afigure}
%        \includegraphics[height=2.5cm, right]{images/ttd.png}
%    \end{afigure}
%    {Lingga Rohadyan}
%\end{flushright}