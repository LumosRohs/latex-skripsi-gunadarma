\chapter{PENDAHULUAN}

\section{Latar Belakang}

\section{Rumusan Masalah}
\begin{enumerate}
    \item Rumusan Masalah 1
    \item Rumuhas Masalah 2
\end{enumerate}

\section{Batasan Masalah}
\begin{enumerate}
    \item Batasan Masalah 1
    \item Batasan Masalah 2
\end{enumerate}

\section{Tujuan Penelitian}
\begin{enumerate}
    \item Tujuan Penelitian 1
    \item Tujuan Penelitian 2
    \item Tujuan Penelitian 3
\end{enumerate}

\section{Metode Penelitian}
\begin{enumerate}
    \item \textbf{Metode Penelitian 1}
    \item[] Deskripsi
    \item \textbf{Metode Penelitian 2}
    \item[] Deskripsi
    \item \textbf{Metode Penelitian 3}
    \item[] Deskripsi
\end{enumerate}

\section{Sistematika Penulisan}
Sistematika pada penulisan ini dibagi ke dalam lima bab yang akan dituliskan berurut, dimana setiap bab memiliki prosedur-prosedur yang akan dibahas, yaitu:

\begin{itemize}[label={\phantom{•}}]
    \item \textbf{BAB I PENDAHULUAN}
    \item[] Pada bab pendahuluan terdapat informasi mengenai, latar belakang masalah, rumusan masalah, batasan masalah, tujuan penelitian, metode penelitian dan sistematika penulisan.
    \item \textbf{BAB II TINJAUAN PUSTAKA}
    \item[] Pada bab ini berisi informasi mengenai teori-teori pendukung yang menjadi sumber informasi pada penelitian ini, selain itu terdapat penjelasan-penjelasan singkat mengenai \textit{software} pendukung penelitian.
    \item \textbf{BAB III METODE PENELITIAN}
    \item[] Bab ini berisi penjelasan langkah-langkah dalam penelitian dan gambaran umum dalam membangun model \textit{deep learning} dari penelitian serta penjelasan kode yang digunakan.
    \item \textbf{BAB IV HASIL DAN PEMBAHASAN}
    \item[] Bab ini merupakan bagian yang membahas tentang hasil pengolahan data dan analisa pada model yang telah dirancang serta hasil uji coba dari program.
    \item \textbf{BAB V PENUTUP}
    \item[] Pada bab penutup meliputi kesimpulan dan saran, kesimpulan akan berkaitan mengenai keberhasilan penelitian terhadap tujuan penelitian yang dilakukan serta saran-saran berisi informasi langkah yang mungkin dapat menjadi penyempurnaan pada penelitian yang dilakukan.
\end{itemize}
