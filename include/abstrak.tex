\chapter*{ABSTRAK}
\begin{singlespace}
    \noindent
    Lingga Rohadyan, 10120594 \\ \\
    \textbf{IDENTIFIKASI NILAI NUTRISI PADA MAKANAN POPULER DI INDONESIA BERBASIS CITRA MENGGUNAKAN CNN DENGAN ARSITEKTUR EFFICIENTNETV2}\\
    Skripsi, Jurusan Sistem Informasi, Fakultas Ilmu Komputer dan Teknologi Informasi, Universitas Gunadarma, 2024.\\
    Kata Kunci : Convolutional Neural Network, Transfer Learning, Klasifikasi.\\
    (xiv + 98 + Lampiran) \\ \\
    Teknologi berkembang sangat pesat dan memberikan dampak yang besar bagi kehidupan manusia. Salah satu aspek paling menarik dari perkembangan teknologi saat ini adalah \textit{Artificial Intelligence} (AI). Dalam perkembangan AI, salah satu cabang yang paling berpengaruh adalah \textit{Machine Learning} (ML). Salah satu penerapan ML pada bidang kesehatan yaitu dengan menggunakan teknologi pengenalan gambar untuk mengklasifikasikan makanan dan mengestimasi kandungan nutrisinya. Salah satu teknik ML yang untuk analisis gambar adalah \textit{Convolutional Neural Network} (CNN).
    Penelitian ini bertujuan untuk menghasilkan model deep learning berbasis arsitektur EfficientNetV2 yang dapat mengklasifikasikan 10 jenis citra makanan dan mengetahui nilai nutrisinya. Dari penelitian ini dapat disimpulkan bahwa model yang dibuat telah berhasil mengklasifikasikan citra makanan dan menampilkan estimasi nutrisinya. HPC DGX A100 terbukti dapat melakukan pelatihan model 2x lebih cepat dari GPU T4. Model yang dilatih menggunakan GPU T4 dengan model yang dilatih menggunakan HPC DGX A100 mempunyai performa dan akurasi yang hampir sama dan tidak berbeda secara signifikan. Pengujian model menggunakan GPU T4 berhasil mendapatkan accuracy 87,50\% pada data validasi, dengan rata-rata \textit{precision}, \textit{recall}, dan \textit{F1-Score} 88\%. Kemudian pengujian model menggunakan HPC DGX A100 berhasil mendapatkan \textit{accuracy} 87,25\% pada data validasi, dengan \textit{precision} 87\%, \textit{recall} 88\%, dan \textit{F1-Score} 87\%. \\ \\
    Daftar Pustaka (2017 - 2024)
\end{singlespace}
\pagebreak